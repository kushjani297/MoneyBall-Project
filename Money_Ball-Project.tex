\PassOptionsToPackage{unicode=true}{hyperref} % options for packages loaded elsewhere
\PassOptionsToPackage{hyphens}{url}
%
\documentclass[
]{article}
\usepackage{lmodern}
\usepackage{amssymb,amsmath}
\usepackage{ifxetex,ifluatex}
\ifnum 0\ifxetex 1\fi\ifluatex 1\fi=0 % if pdftex
  \usepackage[T1]{fontenc}
  \usepackage[utf8]{inputenc}
  \usepackage{textcomp} % provides euro and other symbols
\else % if luatex or xelatex
  \usepackage{unicode-math}
  \defaultfontfeatures{Scale=MatchLowercase}
  \defaultfontfeatures[\rmfamily]{Ligatures=TeX,Scale=1}
\fi
% use upquote if available, for straight quotes in verbatim environments
\IfFileExists{upquote.sty}{\usepackage{upquote}}{}
\IfFileExists{microtype.sty}{% use microtype if available
  \usepackage[]{microtype}
  \UseMicrotypeSet[protrusion]{basicmath} % disable protrusion for tt fonts
}{}
\makeatletter
\@ifundefined{KOMAClassName}{% if non-KOMA class
  \IfFileExists{parskip.sty}{%
    \usepackage{parskip}
  }{% else
    \setlength{\parindent}{0pt}
    \setlength{\parskip}{6pt plus 2pt minus 1pt}}
}{% if KOMA class
  \KOMAoptions{parskip=half}}
\makeatother
\usepackage{xcolor}
\IfFileExists{xurl.sty}{\usepackage{xurl}}{} % add URL line breaks if available
\IfFileExists{bookmark.sty}{\usepackage{bookmark}}{\usepackage{hyperref}}
\hypersetup{
  pdftitle={Money-Ball Project},
  pdfauthor={Kush Jani},
  pdfborder={0 0 0},
  breaklinks=true}
\urlstyle{same}  % don't use monospace font for urls
\usepackage[margin=1in]{geometry}
\usepackage{graphicx,grffile}
\makeatletter
\def\maxwidth{\ifdim\Gin@nat@width>\linewidth\linewidth\else\Gin@nat@width\fi}
\def\maxheight{\ifdim\Gin@nat@height>\textheight\textheight\else\Gin@nat@height\fi}
\makeatother
% Scale images if necessary, so that they will not overflow the page
% margins by default, and it is still possible to overwrite the defaults
% using explicit options in \includegraphics[width, height, ...]{}
\setkeys{Gin}{width=\maxwidth,height=\maxheight,keepaspectratio}
\setlength{\emergencystretch}{3em}  % prevent overfull lines
\providecommand{\tightlist}{%
  \setlength{\itemsep}{0pt}\setlength{\parskip}{0pt}}
\setcounter{secnumdepth}{-2}
% Redefines (sub)paragraphs to behave more like sections
\ifx\paragraph\undefined\else
  \let\oldparagraph\paragraph
  \renewcommand{\paragraph}[1]{\oldparagraph{#1}\mbox{}}
\fi
\ifx\subparagraph\undefined\else
  \let\oldsubparagraph\subparagraph
  \renewcommand{\subparagraph}[1]{\oldsubparagraph{#1}\mbox{}}
\fi

% set default figure placement to htbp
\makeatletter
\def\fps@figure{htbp}
\makeatother


\title{Money-Ball Project}
\author{Kush Jani}
\date{6/19/2020}

\begin{document}
\maketitle

\hypertarget{lets-get-started}{%
\subsection{Let's get started!}\label{lets-get-started}}

Follow the steps outlined in bold below using your new R skills and help
the Oakland A's recruit under-valued players!

\textbf{Use R to open the Batting.csv file and assign it to a dataframe
called batting using read.csv}
\texttt{here\ I\ mentioned\ my\ path\ in\ code\ ,\ You\ have\ to\ write\ your\ path\ as\ you\ have\ saved\ your\ fie\ in\ your\ PC}

\begin{verbatim}
batting <- read.csv('C:/Users/Kush/Documents/R/R-Course-HTML-Notes/R-Course-HTML-Notes/R-for-Data-Science-and-Machine-Learning/Training Exercises/Capstone and Data Viz Projects/Capstone Project/Batting.csv')
\end{verbatim}

\textbf{Use head() to check out the batting}

\begin{verbatim}
head(batting)
\end{verbatim}

\textbf{Use str() to check the structure. Pay close attention to how
columns that start with a number get an `X' in front of them! You'll
need to know this to call those columns!}

\begin{verbatim}
str(batting)
\end{verbatim}

\textbf{Call the head() of the first five rows of AB (At Bats) column}

\begin{verbatim}
head(batting$AB)
\end{verbatim}

\textbf{Call the head of the doubles (X2B) column}

\begin{verbatim}
head(batting$X2B)
\end{verbatim}

\hypertarget{feature-engineering}{%
\subsection{Feature Engineering}\label{feature-engineering}}

\textbf{We need to add three more statistics that were used in
Moneyball! These are}

\texttt{Batting\ Average} \texttt{On\ Base\ Percentage}
\texttt{Slugging\ Percentage}

\textbf{Batting Average is equal to H (Hits) divided by AB (At Base). So
we'll do the following to create a new column called BA and add it to
our data frame:}

\begin{verbatim}
batting$BA <- batting$H / batting$AB
\end{verbatim}

\textbf{After doing this operation, check the last 5 entries of the BA
column of your data frame and it should look like this:}

\begin{verbatim}
tail(batting$BA,5)
\end{verbatim}

\textbf{Now do the same for some new columns! On Base Percentage (OBP)
and Slugging Percentage (SLG)}
\texttt{HINT:-\ For\ SLG,\ you\ need\ 1B\ (Singles),\ this\ isn\textquotesingle{}t\ in\ your\ data\ frame.}
\texttt{However\ you\ can\ calculate\ it\ by\ subtracting\ doubles,triples,\ and\ home\ runs\ from\ total\ hits\ (H):\ 1B\ =\ H-2B-3B-HR}

\textbf{Create an OBP Column \& Create an SLG Column} \textbf{On Base
Percentage}

\begin{verbatim}
batting$OBP <- (batting$H + batting$BB + batting$HBP)/(batting$AB + batting$BB + batting$HBP + batting$SF)
\end{verbatim}

\textbf{Creating X1B (Singles)}

\begin{verbatim}
batting$X1B <- batting$H - batting$X2B - batting$X3B - batting$HR
\end{verbatim}

\textbf{Creating Slugging Average (SLG)}

\begin{verbatim}
batting$SLG <- ((1 * batting$X1B) + (2 * batting$X2B) + (3 * batting$X3B) + (4 * batting$HR) ) / batting$AB
\end{verbatim}

\textbf{Check the structure of your data frame using str()}

\begin{verbatim}
str(batting)
\end{verbatim}

\hypertarget{merging-salary-data-with-batting-data}{%
\subsection{Merging Salary Data with Batting
Data}\label{merging-salary-data-with-batting-data}}

\texttt{We\ know\ we\ don\textquotesingle{}t\ just\ want\ the\ best\ players,\ we\ want\ the\ most\ undervalued\ players,}
\texttt{meaning\ we\ will\ also\ need\ to\ know\ current\ salary\ information!\ We\ have\ salary\ information\ in\ the\ csv\ file\ \textquotesingle{}Salaries.csv\textquotesingle{}.}

\textbf{Complete the following steps to merge the salary data with the
player stats!}

\textbf{Load the Salaries.csv file into a dataframe called sal using
read.csv}

\begin{verbatim}
sal <- read.csv('C:/Users/Kush/Documents/R/R-Course-HTML-Notes/R-Course-HTML-Notes/R-for-Data-Science-and-Machine-Learning/Training Exercises/Capstone and Data Viz Projects/Capstone Project/Salaries.csv')
\end{verbatim}

\textbf{Use summary to get a summary of the batting data frame and
notice the minimum year in the yearID column. Our batting data goes back
to 1871! Our salary data starts at 1985, meaning we need to remove the
batting data that occured before 1985 \& Use subset() to reassign
batting to only contain data from 1985 and onwards}

\begin{verbatim}
summary(batting)
batting <- subset(batting,yearID >= 1985)
\end{verbatim}

\textbf{Now use summary again to make sure the subset reassignment
worked, your yearID min should be 1985}

\begin{verbatim}
summary(batting)
\end{verbatim}

\textbf{Use the merge() function to merge the batting and sal data
frames by c(`playerID',`yearID'). Call the new data frame combo}

\begin{verbatim}
combo <- merge(batting,sal,by=c('playerID','yearID'))
\end{verbatim}

\textbf{Use summary to check the data}

\begin{verbatim}
summary(combo)
\end{verbatim}

\hypertarget{analyzing-the-lost-players}{%
\subsection{Analyzing the Lost
Players}\label{analyzing-the-lost-players}}

\texttt{As\ previously\ mentioned,\ the\ Oakland\ A\textquotesingle{}s\ lost\ 3\ key\ players\ during\ the\ off-season.}
\texttt{We\textquotesingle{}ll\ want\ to\ get\ their\ stats\ to\ see\ what\ we\ have\ to\ replace.}
\texttt{The\ players\ lost\ were:\ first\ baseman\ 2000\ AL\ MVP\ Jason\ Giambi\ (giambja01)\ to\ the\ New\ York\ Yankees,}
\texttt{outfielder\ Johnny\ Damon\ (damonjo01)\ to\ the\ Boston\ Red\ Sox\ and\ infielder\ Rainer\ Gustavo\ "Ray"\ Olmedo\ (\textquotesingle{}saenzol01\textquotesingle{}).}

\textbf{Use the subset() function to get a data frame called
lost\_players from the combo data frame consisting of those 3 players.
Hint: Try to figure out how to use \%in\% to avoid a bunch of or
statements!}

\begin{verbatim}
lost_players <- subset(combo,playerID %in% c('giambja01','damonjo01','saenzol01') )
lost_players
\end{verbatim}

\texttt{Since\ all\ these\ players\ were\ lost\ in\ after\ 2001\ in\ the\ offseason,\ let\textquotesingle{}s\ only\ concern\ ourselves\ with\ the\ data\ from\ 2001.}

\textbf{Use subset again to only grab the rows where the yearID was
2001.}

\begin{verbatim}
lost_players <- subset(lost_players,yearID == 2001)
\end{verbatim}

\textbf{Reduce the lost\_players data frame to the following columns:
playerID,H,X2B,X3B,HR,OBP,SLG,BA,AB}

\begin{verbatim}
lost_players <- lost_players[,c('playerID','H','X2B','X3B','HR','OBP','SLG','BA','AB')]
head(lost_players)
\end{verbatim}

\hypertarget{replacement-players}{%
\subsection{Replacement Players}\label{replacement-players}}

\texttt{Now\ we\ have\ all\ the\ information\ we\ need!\ Here\ is\ your\ final\ task\ -\ Find\ Replacement\ Players\ for\ the\ key\ three\ players\ we\ lost!}
\textbf{However, you have three constraints: (1)The total combined
salary of the three players can not exceed 15 million dollars.}
\textbf{(2)Their combined number of At Bats (AB) needs to be equal to or
greater than the lost players.} \textbf{(3)Their mean OBP had to equal
to or greater than the mean OBP of the lost players}

\hypertarget{example-solution}{%
\subsection{Example Solution}\label{example-solution}}

\texttt{Note:\ There\ are\ lots\ of\ correct\ answers\ and\ ways\ to\ solve\ this!}

\textbf{First only grab available players from year 2001}

\begin{verbatim}
library(dplyr)
avail.players <- filter(combo,yearID==2001)
\end{verbatim}

\textbf{Then I made a quick plot to see where I should cut-off for
salary in respect to OBP:}

\begin{verbatim}
library(ggplot2)
ggplot(avail.players,aes(x=OBP,y=salary)) + geom_point()
\end{verbatim}

\textbf{Looks like there is no point in paying above 8 million or so
(I'm just eyeballing this number). I'll choose that as a cutt off point.
There are also a lot of players with OBP==0. Let's get rid of them too.}

\begin{verbatim}
avail.players <- filter(avail.players,salary<8000000,OBP>0)
\end{verbatim}

\textbf{The total AB of the lost players is 1469. This is about 1500,
meaning I should probably cut off my avail.players at 1500/3= 500 AB.}

\begin{verbatim}
avail.players <- filter(avail.players,AB >= 500)
\end{verbatim}

\textbf{Now let's sort by OBP and see what we've got!}

\begin{verbatim}
possible <- head(arrange(avail.players,desc(OBP)),10)
\end{verbatim}

\textbf{Grab columns I'm interested in:}

\begin{verbatim}
possible <- possible[,c('playerID','OBP','AB','salary')]
possible
\end{verbatim}

\textbf{Can't choose giambja again, but the other ones look good (2-4).
I choose them!}

\begin{verbatim}
possible[2:4,]
\end{verbatim}

\textbf{Great, looks like I just saved the 2001 Oakland A's a lot of
money! If only I had a time machine and R, I could have made a lot of
money in 2001 picking players!}

\texttt{Great\ Job!\ Here\ we\ completed\ MoneyBall\ project!}

\hypertarget{the-end}{%
\subsection{The End!}\label{the-end}}

\end{document}
